\documentclass[a4paper]{article}
\setlength{\topmargin}{-1.0in}
\setlength{\oddsidemargin}{-0.2in}
\setlength{\evensidemargin}{0in}
\setlength{\textheight}{10.5in}
\setlength{\textwidth}{6.5in}
\setlength{\parindent}{0pt}
\usepackage{enumitem}
\usepackage{amsmath}
\usepackage{hyperref}
\usepackage{amssymb}
\usepackage{mathtools}
\usepackage{minted}
\usepackage[dvipsnames]{xcolor}
\usepackage{mathpartir}
\newlist{sollist}{itemize}{1}
\setlist[sollist]{label=$\implies$}
\usepackage{algorithm}
\usepackage{algorithmic}
\usepackage{mathtools}
\DeclarePairedDelimiter{\ceil}{\lceil}{\rceil}
\DeclarePairedDelimiter{\floor}{\lfloor}{\rfloor}
\DeclareMathOperator{\Tr}{Tr}
\DeclareMathOperator*{\argmax}{arg\,max}
\DeclareMathOperator*{\argmin}{arg\,min}

\hypersetup{
    colorlinks=true,
    linkcolor=blue,
    filecolor=magenta,      
    urlcolor=cyan,
    pdftitle={Assignment 2},
    pdfpagemode=FullScreen,
    }
\def\endproofmark{$\Box$}
\newcommand{\norm}[1]{\left\lVert#1\right\rVert}
\newenvironment{proof}{\par{\bf Proof}:}{\endproofmark\smallskip}
\begin{document}
\begin{center}
{\large \bf \color{red}  Department of Computer Science} \\
{\large \bf \color{red}  Ashoka University} \\

\vspace{0.1in}

{\large \bf \color{blue} Introduction to Machine Learning: CS-3410-1}

\vspace{0.05in}

    { \bf \color{YellowOrange} Assignment 2}
\end{center}
\medskip

{\textbf{Name: Dhruman Gupta} }

\bigskip
\hrule

\section{Question 1}
Done on jupyter notebook.

\section{Question 2}
Done on jupyter notebook.

\section{Question 3}
Done on jupyter notebook.

Question 4 is done next page onwards.

\newpage

\section{Do you even shatter?}

\subsection{$1\{a < x\}$}

The VC dimension of this hypothesis class is 1. Given an $x \in \mathbb{R}$, the hypothesis $h(x) = 1\{a < x\}$ can shatter the set $\{x\}$ by choosing $a < x$ or $a > x$ depending on the label of $x$.

However, if we have 2 points, such that $x_1 < x_2$, where $x_1$ is labelled $1$ and $x_2$ is labelled $0$, then any choice of $a$ will result in the other point being misclassified. Thus, the VC dimension is 1.

\subsection{$1\{a < x < b\}$}

The VC dimension of this hypothesis class is 2. Given two points $x_1 < x_2$, the hypothesis $h(x) = 1\{a < x < b\}$ can shatter these two points:

\begin{itemize}
    \item For labelling $(1, 1)$, we can choose $a < x_1 < x_2 < b$.
    \item For labelling $(0, 0)$, we can choose $x_1 < a < b < x_2$.
    \item For labelling $(1, 0)$, we can choose $a < x_1 < b < x_2$.
    \item For labelling $(0, 1)$, we can choose $x_1 < a < x_2 < b$.
\end{itemize}


Now, say we have 3 points, $x_1 < x_2 < x_3$. Given the labelling $(1, 0, 1)$, we can see that no choice of $a$ and $b$ will result in all the points being labelled correctly. Thus, the VC dimension is 2.

\subsection{$1\{a \sin(x) > 0\}$}

The VC dimension of this hypothesis class is 1. Pick any $x_1$ with $\sin x_1 \neq 0$. To label $1$, we can choose $a = \frac{\sin x_1}{|\sin x_1|}$. This will label $x_1$ as $1$. If the label is $0$, we can choose $a = -\frac{\sin x_1}{|\sin x_1|}$. This will label $x_1$ as $0$.\\


Now, consider two points. If the signs of $\sin x_1$ and $\sin x_2$ are the same, then the labelling $(1, 0)$ or $(0, 1)$ cannot be achieved. If the signs are different, then the labelling $(1, 1)$ cannot be achieved. Thus, the VC dimension is 1.

\subsection{$1\{\sin (x + a) > 0\}$}

The VC dimension of this hypothesis class is 2. Given two points $x_1 < x_2$, the hypothesis $h(x) = 1\{\sin (x + a) > 0\}$ can shatter these two points:

\begin{itemize}
    \item For labelling $(1, 1)$, we can choose $a$ such that $\sin (x_1 + a) > 0$ and $\sin (x_2 + a) > 0$.
    \item For labelling $(0, 0)$, we can choose $a$ such that $\sin (x_1 + a) < 0$ and $\sin (x_2 + a) < 0$.
    \item For labelling $(1, 0)$, we can choose $a$ such that $\sin (x_1 + a) > 0$ and $\sin (x_2 + a) < 0$.
    \item For labelling $(0, 1)$, we can choose $a$ such that $\sin (x_1 + a) < 0$ and $\sin (x_2 + a) > 0$.
\end{itemize}

As long as $x_1 - x_2 \mod \pi \equiv 0$ is not true, we can always find an $a$ such that the labelling is achieved. Since for any $x_1, x_2$ where the above is not true, we can shatter them, the VC dimension is 2.\\

Consider 3 points, $x_1, x_2, x_3$. Calculate $\theta_1, \theta_2, \theta_3$ as $\theta_i = x_i \mod 2\pi$. Now, WLOG, assume $\theta_1 \leq \theta_2 \leq \theta_3$. With this ordering (which will be there for every single point), the labelling $(1, 0, 1)$ cannot be achieved. Because if $\sin(a + x_1) > 0$, and $\sin(a + x_3) > 0$, then it must be the case that $\sin(a + x_2) > 0$ too. This comes from the fact that $\sin$ is continous, and $x_2$ is sandwiched between $x_1$ and $x_3$. So, $\sin x_2$ cannot be lesser than $0$. Thus, the VC dimension is 2.

\newpage

\section{Question 5}

In this case, we can infact shatter all 3 options (I almost tried with a compass to verify this).\\

What this tells me is that the VC dimension is at least 4 - as it can shatter a set of 4 points. 

\end{document}